\documentclass[12pt,twoside]{article}


\usepackage{graphicx}
\usepackage{caption}
\usepackage{jmlda}

\begin{document}
\title
    {Автоматическое дополнение плейлистов в рекомендательной системе пользователей}
\author
    {Кислинский~В.\,Г., Фролов~Е.\,, Воронцов~К.\,В.} % основной список авторов, выводимый в оглавление
\email
    {kislinskiy.vg@phystech.edu; evgeny.frolov@skolkovotech.ru; vokov@forecsys.ru}
\thanks
    {Работа выполнена при финансовой поддержке РФФИ, проект \No\,00-00-00000.
     Научный руководитель:  Воронцов~К.\,В.
     Консультант:  Фролов~Е.}

\organization
    {Московский физико-технический институт}
\abstract
	{Работа посвящена исследованию метода совместной матричной факторизации в задаче top-N рекомендаций для автоматического продолжения плейлистов. Предлагается  модель матричной факторизации, учитывающий дополнительную информацию о плейлистах и треках. Данный метод будет иметь, не только преимущество алгоритмов коллаборативной фильтрации, которые способны выявить скрытые свойства пользователей и обьектов,  но также сможеть учитывать контекстную информацию, что поможет решить проблему холодного старта для обьектов. В данном методе будет введена дополнительная регуляризация, основанная на предположение, что если обьекты близки в пространстве признаков, то они близки в латентном факторном пространстве. Для анализа качества представленного алгоритма проводятся эксперименты на выборке  из миллиона плейлистов MPD. 

\bigskip
\textbf{Ключевые слова}: \emph {задача top-N рекомендаций, совместная матричная факторизация, алгоритм LCE, латентное факторное пространство, коллаборативная фильтрация}.

}

\maketitle

\section{1 Введение}

Большинство методов коллаборативной фильтрации имеют ряд недостатков, основным из которых является проблема холодного старта. Другой подход к задаче рекомендаций, основанный на дополнительный информации, не имеет этой проблемы. 

\section{2 Постановка задачи}


\section{3 Базовый метод}

\subsection{3.1 }

\subsection{3.2 }

\subsection{3.3 }

\section{4}

\section{5 Базовый вычислительный эксперимент}

	
\bibliography{}
\bibliographystyle{unsrt}

\end{document}
