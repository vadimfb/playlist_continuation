\documentclass{beamer}
\usepackage[utf8]{inputenc}
\usepackage[russian]{babel}
\usepackage{amsmath,mathrsfs,mathtext}
\usepackage{graphicx, epsfig}
\usepackage{amsmath}
\DeclareMathOperator*{\argmin}{arg\,min}
\DeclareMathOperator*{\argmax}{arg\,max}
\usetheme{Warsaw}%{Singapore}%{Warsaw}%{Warsaw}%{Darmstadt}
\usecolortheme{sidebartab}
\definecolor{beamer@blendedblue}{RGB}{21,96,189}
%----------------------------------------------------------------------------------------------------------
\title[\hbox to 56mm{Автоматическое дополнение плейлистов  \hfill\insertframenumber\,/\,\inserttotalframenumber}]
{Автоматическое дополнение музыкальных плейлистов \\ в рекомендательной системе}
\author[В.\,Г. Кислинский]{\large \\Кислинский Вадим Геннадьевич}
\institute{\large
Московский физико-технический институт}

\date{\footnotesize{\emph{Курс:} Численные методы обучения по прецедентам\par (практика, В.\,В. Стрижов)/Группа 574, весна 2018}}
%----------------------------------------------------------------------------------------------------------
\begin{document}
%----------------------------------------------------------------------------------------------------------
\begin{frame}
%\thispagestyle{empty}
\titlepage
\end{frame}
%-----------------------------------------------------------------------------------------------------
\begin{frame}{Цель исследование}

\begin{block}{Цель}
Исследовать метод матричной факторизации в задаче автоматического дополнения плейлистов
\end{block}

\begin{block}{Проблемы}

\begin{itemize}
\item{Классический метод матричной факторизации не учитывает дополнительную информацию о плейлистах и треках}
\item{Проблема холодного старта для новых плейлистов}
\end{itemize}
\end{block}

\begin{block}{Было предложено}
Использовать метод совместной матричной факторизации
\end{block}
\end{frame}
%----------------------------------------------------------------------------------------------------------
\begin{frame}{Литература}

{\bf Обзор основных подходов к задаче автоматического дополнения плейлистов}
\begin{enumerate}
\item[1]Geoffray Bonnin, Dietmar Jannach. Automated Generation of Music Playlists: Survey and
Experiments. ACM Computing Surveys (CSUR). 2014
\end{enumerate}
{\bf Основные методы в задаче top-{N} рекомендаций}
\begin{enumerate}
\item[2] Paolo Cremonesi al. \textit{Performance of recommender algorithms on top-n recommendation tasks.} 2010
\end{enumerate}
{\bf Метод совместной факторизации}
\begin{enumerate}
\item[3] 	Dimitrios Rafailidis al. \textit{Modeling the Dynamics of User Preferences in Coupled Tensor Factorization.}  2013
\end{enumerate}
{\bf Алгоритм LCE}
\begin{enumerate}
\item[4]Martin Saveski al. \textit{ Item Cold-Start Recommendations:
Learning Local Collective Embeddings.} 2014
\end{enumerate}
\end{frame}
%----------------------------------------------------------------------------------------------------------
\begin{frame}{Постановка задачи}
\begin{block}{Дано}
$\mathcal{P}$ - множество из ${n}$ плейлистов, $\mathcal{T}$ - множество из ${m}$ треков, матрица $\mathbf{R} \in \mathbb{R}^{n \times m}$, где $\mathbf{R}_{ij} = 1$, 
если $t_j \in {p}_j$. Каждый трек описывается автором и альбомом, плейлист названием. Эта информация задается матрицей $\mathbf{X} \in \mathbb{R}^{n \times (k + d + l)}$, первые $k$ столбцов показывают, какие авторы входят в плейлист, следующие $d$, какие альбомы, последние $l$ описывают названия плейлистов
\end{block}
\begin{block}{Задача}
Для нового плейлиста построить вектор $\mathbf{r} \in \mathbb{R}^{m \times 1}$, $i$-ый элемент которого означает насколько трек $t_i$ подходит данному плейлисту
\end{block}
\end{frame}
%----------------------------------------------------------------------------------------------------------
\begin{frame}{Задача совместной матричной факторизации}

Ищется разложение матрицы $\mathbf{R}$ на две матрицы меньшей размерности \[ \mathbf{R} \approx \mathbf{UV}\]

Предполагая, что профили плейлистов зависят от того, какие исполнители, альбомы входят в плейлист, 
какие названия у плейлистов, запишем \[\mathbf{X} \approx \mathbf{UH}\]

Приходим к задаче оптимизации:

\begin{equation}
\begin{gathered}
\hat{\mathbf{U}}, \hat{\mathbf{V}}, \hat{\mathbf{H}} = \argmin_{\mathbf{U, V, H}}\alpha||\mathbf{R} - \mathbf{UV}||_F^2 +(1 - \alpha) ||\mathbf{X} - \mathbf{UH}||_F^2 + \\
 \lambda(||\mathbf{U}||_F^2 + ||\mathbf{V}||_F^2 + ||\mathbf{H}||_F^2), \\
 \text{s.t.}  \quad \mathbf{U} \geq 0,  \mathbf{V} \geq 0,   \mathbf{H} \geq 0
\end{gathered}
\end{equation}

\end{frame}

%------------------------------------------------------------------------------------------------------------
\begin{frame}{Введение дополнительной регуляризации на основе близости плейлистов в пространстве признаков}

Составим граф близости плейлистов в котором вершинами будут плейлисты. С помощью матрицы смежности ~--- $\mathbf{A}$ определим близость, соответственных профилей:

\begin{equation}
\begin{gathered}
S = \frac{1}{2}\sum_{i, j = 1}^n||\mathbf{u}_i - \mathbf{u}_j||^2\mathbf{A}_{ij} = \sum_{i = 1}^n (\mathbf{u}_i^{T}\mathbf{u}_i)\mathbf{D}_{ii} - \sum_{i,j = 1}^n (\mathbf{u}_i^{T}\mathbf{u}_j)\mathbf{A}_{ij} =\\= \text{Tr}(\mathbf{U}^{T}\mathbf{DU}) -  \text{Tr}(\mathbf{U}^{T}\mathbf{AU}) = \text{Tr}(\mathbf{U}^{T}\mathbf{LU}),
\end{gathered}
\end{equation}

Перепишем (1) с учетом $S$

\begin{equation}
\begin{gathered}
\hat{\mathbf{U}}, \hat{\mathbf{V}}, \hat{\mathbf{H}} = \argmin_{\mathbf{U, V, H}}\alpha||\mathbf{R} - \mathbf{UV}||_F^2 +(1 - \alpha) ||\mathbf{X} - \mathbf{UH}||_F^2 + \\ \beta\text{Tr}(\mathbf{U}^T\mathbf{LU}) +  \lambda(||\mathbf{U}||_F^2 + ||\mathbf{V}||_F^2 + ||\mathbf{H}||_F^2),\\
\text{s.t.} \quad \mathbf{U} \geq 0,  \mathbf{V} \geq 0,  \mathbf{H} \geq 0.
\end{gathered}
\end{equation}

\end{frame}
%----------------------------------------------------------------------------------------------------------
\begin{frame}{Алгоритм поиска стационарной точки}

Матрицы $(\mathbf{U, V, H})$ обновляются по следующим правилам:

\begin{itemize}
\item$\mathbf{U} = \mathbf{U} \odot \dfrac{\alpha\mathbf{R}\mathbf{V}^T + (1 -\alpha)\mathbf{X}\mathbf{H}^T + \beta\mathbf{AU}}{\alpha\mathbf{UV}\mathbf{V}^T + (1 - \alpha)\mathbf{UH}\mathbf{H}^T + \beta\mathbf{DU} + \lambda\mathbf{U}}$

\item$\mathbf{V} = \mathbf{V} \odot \dfrac{\alpha\mathbf{U}^T\mathbf{R}}{\alpha\mathbf{U}^T\mathbf{UV} + \lambda\mathbf{V}}$

\item$\mathbf{H} = \mathbf{H} \odot \dfrac{(1 - \alpha)\mathbf{U}^T\mathbf{X}}{(1 - \alpha)\mathbf{U}^T\mathbf{UH} + \lambda\mathbf{H}}$
\end{itemize}

\end{frame}

\begin{frame}{Получение рекомендаций}

Для нового плейлиста составим строку признакового описания $\mathbf{x} \in \mathbb{R}^{1 \times  (k + d + r)}$.  С помощью метода наименьших квадратов из системы \[\mathbf{x} = \mathbf{u}\mathbf{H}\] найдем профиль плейлиста $\mathbf{u}$. \[\mathbf{r} = \mathbf{uV},\] где $i$-ый элемент вектора $\mathbf{r}$ означает насколько трек $t_i$ подходит новому плейлисту , после этого из вектора $\mathbf{r}$ выбирается top-$N$ значений с индексами $\{i_1, \ldots, i_N\}$ и рекомендуются треки~$\{t_{i_1}, \ldots, t_{i_N}\}$.

\end{frame}
%-------------------------------------------------------------------------------------------------------
\begin{frame}{Критерии качества}

Пусть $R$ список рекомендованных треков, $G$ - список истинных треков.

$$\text{R-precision} = \frac{|G \cap R_{1:|G|}|}{|G|}$$
$$DCG = 1 + \sum_{i = 2}^{|R|} \frac{rel_i}{\log_{2}i}$$
$$IDCG = 1 + \sum_{i = 2}^{|R|} \frac{1}{\log_{2}i}$$
$$nDCG = \frac{DCG}{IDCG}$$

\end{frame}
%-----------------------------------------------------------------------------------------------------------------------------------------------------------------
\begin{frame}{Эксперимент}

\begin{block}{Данные}
Cлучайная подвыборка из миллиона плейлистов, содержащая 7657 плейлистов и 8560 треков, при этом каждый плейлист содержит не менее пяти треков. 
\end{block}

\begin{block}{Базовый алгоритм - PureSVD}
$$\mathbf{R} = \mathbf{U\Sigma Q^T}$$
$$\mathbf{R} = \mathbf{PQ^T},  \mathbf{P} = \mathbf{U\Sigma} = \mathbf{RQ}$$
$$\mathbf{\hat{r}_u} = \mathbf{r_uQQ^T}$$
\end{block}

\end{frame}
%-----------------------------------------------------------------------------------------------------------------------------------------------------------------
\begin{frame}{Результаты эксперимента}
\begin{figure}[ht]\center
\includegraphics[width=1\textwidth]{recall.pdf}\\
\caption{Зависимость качества от ранга разложения.}
\end{figure}
\end{frame}
%-----------------------------------------------------------------------------------------------------------------------------------------------------------------
\begin{frame}{Результаты эксперимента}
\begin{figure}[ht]\center
\includegraphics[width=1\textwidth]{recall2.pdf}\\
\caption{Зависимость качества от topk.}
\end{figure}
\end{frame}
%-----------------------------------------------------------------------------------------------------------------------------------------------------------------
\begin{frame}{Результаты эксперимента}
\begin{figure}[ht]\center
\includegraphics[width=1\textwidth]{nDCG.pdf}\\
\caption{Зависимость качества от ранга разложения.}
\end{figure}
\end{frame}
%-----------------------------------------------------------------------------------------------------------------------------------------------------------------
\begin{frame}{Результаты эксперимента}
\begin{figure}[ht]\center
\includegraphics[width=1\textwidth]{nDCG2.pdf}\\
\caption{Зависимость качества от  topk.}
\end{figure}
\end{frame}
%-----------------------------------------------------------------------------------------------------------------------------------------------------------------
\begin{frame}{Заключение}
\begin{itemize}
\item Исследован метод матричной матричной факторизации в задаче автоматического дополнения плейлистов.
\item Исследовано влияние учета дополнительной информации в методе матричной факторизации.
\item Проверено предположение о близости плейлистов в латентном пространстве.
\end{itemize}
\end{frame}

\end{document}