\documentclass{beamer}
\usepackage[utf8]{inputenc}
\usepackage[russian]{babel}
\usepackage{amsmath,mathrsfs,mathtext}
\usepackage{graphicx, epsfig}
\usetheme{Warsaw}%{Singapore}%{Warsaw}%{Warsaw}%{Darmstadt}
\usecolortheme{sidebartab}
\definecolor{beamer@blendedblue}{RGB}{21,96,189}
%----------------------------------------------------------------------------------------------------------
\title[\hbox to 56mm{Автоматическое дополнение плейлистов  \hfill\insertframenumber\,/\,\inserttotalframenumber}]
{Автоматическое дополнение плейлистов \\ в рекомендательной системе пользователей}
\author[В.\,Г. Кислинский]{\large \\Кислинский Вадим Геннадьевич}
\institute{\large
Московский физико-технический институт}

\date{\footnotesize{\emph{Курс:} Численные методы обучения по прецедентам\par (практика, В.\,В. Стрижов)/Группа 574, весна 2018}}
%----------------------------------------------------------------------------------------------------------
\begin{document}
%----------------------------------------------------------------------------------------------------------
\begin{frame}
%\thispagestyle{empty}
\titlepage
\end{frame}
%-----------------------------------------------------------------------------------------------------
\begin{frame}{Цель исследование}

\begin{block}{Цель}
Решить задачу top-N рекомендаций
\end{block}

\begin{block}{Метод}
Матричная факторизация, учитывающая дополнительные знания о плейлистах и треках
\end{block}
\end{frame}
%----------------------------------------------------------------------------------------------------------
\begin{frame}{Литература}

{\bf Обзор классических методов}
\begin{enumerate}
\item[1] Paolo Cremonesi al. \textit{Performance of recommender algorithms on top-n recommendation tasks.} 2010
\end{enumerate}
{\bf Метод совместной факторизации}
\begin{enumerate}
\item[2] 	Dimitrios Rafailidis al. \textit{Modeling the Dynamics of User Preferences in Coupled Tensor Factorization.}  2013
\end{enumerate}
{\bf Алгоритм LCE}
\begin{enumerate}
\item[3]Martin Saveski al. \textit{ Item Cold-Start Recommendations:
Learning Local Collective Embeddings.} 2014
\end{enumerate}
\end{frame}
%----------------------------------------------------------------------------------------------------------
\begin{frame}{Постановка задачи}
\begin{block}{Дано}
\begin{enumerate}
\item $\math{U}$ - множество из $\math{n}$ плейлистов, 
\item $\math{I}$ - множество из $\math{m}$ треков,
\item $\math{D} = \{(u, i) | u \in U, i \in I \}$ - множество транзакций,
\item $\math{R}$ - матрица $\math{n \times m}$, где $\math{R_{ui}} = 1$, 
если $\math{(u, i) \in D}$,
\item $\math{X_U}$ - матрица $\math{n \times v}$ признакового описания плейлистов,
\item $\math{X_I}$ - матрица $\math{m \times w}$ признакового описания треков
\end{enumerate}

\end{block}
\begin{block}{Задача}
Для плейлиста  $\math{u}$ построить вектор $\mathbf{r}$  из $\math{m}$ элементов, которые означают насколько треки подходят данному плейлисту
\end{block}


\end{frame}
%----------------------------------------------------------------------------------------------------------
\begin{frame}{Постановка задачи}
\begin{block}{Оптимизационная задача}

\begin{center}
$$\arg\min_{\mathbf{W, H_I, H_U}}(\alpha||\mathbf{X_U} - \mathbf{WH_U}||^2 + (1 - \alpha)||\mathbf{R} - \mathbf{WH_I}||^2 + $$ $$ +  \lambda(||\mathbf{W}||^2_F + ||\mathbf{H_U}||^2_F + ||\mathbf{H_I}||^2_F))$$ $$ s. t. \mathbf{W} \geq 0, \mathbf{H_U} \geq 0, 
\mathbf{H_I} \geq 0
\end{center}
\end{block}
\begin{block}{Вычисляем $\mathbf{r}$ для пользователя $\math{u}$}
Пусть $\mathbf{x}$ - признаковое описание плейлиста.

Решим систему $\mathbf{x} = \mathbf{H_U^T}\mathbf{w}$ относительно $\mathbf{w}$,
и определим $$\mathbf{r} = \mathbf{H_I^Tw}$$
\end{block}
\end{frame}
%----------------------------------------------------------------------------------------------------------
\begin{frame}{Постановка задачи}
\begin{block}{Метрики качества}

$\math{R}$ - список top-N рекомендаций, $\math{G}$ - список настоящих треков плейлиста 

$$R@call = \frac{|G \cap R_{1:|G|}|}{|G|}$$
$$Presicion = \frac{|G \cap R_{1:|G|}|}{|R_{1:|G|}|}$$
\end{block}
\end{frame}
%-------------------------------------------------------------------------------------------------------
\begin{frame}{Эксперимент}
\begin{block}{Данные}
Выборка из 20000 плейлистов, содержащая 265464 различных треков, количество транзакций - 1302790.
\end{block}

\begin{block}{Базовый алгоритм - PureSVD}
Неизвестные значения матрицы $\mathbf{R}$ заполняются нулями и делается SVD разложение, полученной матрицы.
$$\mathbf{\hat{R}} = \mathbf{U\Sigma Q^T}$$
$$\mathbf{R} = \mathbf{PQ^T},  \mathbf{P} = \mathbf{U\Sigma} = \mathbf{RQ}$$
$$\mathbf{\hat{r}_u} = \mathbf{r_uQQ^T}$$
  
\end{block}

\end{frame}

\begin{frame}{Результаты эксперимента}
\begin{figure}[h]
\begin{minipage}[h]{0.49\linewidth}
\center{\includegraphics[width=1\linewidth]{recall.png}}
\end{minipage}
\begin{minipage}[h]{0.49\linewidth}
\center{\includegraphics[width=1\linewidth]{precision.png}}
\end{minipage}
\caption{Зависимость метрики от ранга разложения.}
\label{ris:image1}
\end{figure}
\end{frame}

\end{document}